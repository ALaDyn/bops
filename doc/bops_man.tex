\documentclass[11pt]{article}
\setlength{\textheight}{21.80cm}
\setlength{\textwidth}{15.20cm}
\setlength{\oddsidemargin}{0cm}
\usepackage{graphicx}
\usepackage{epsfig}
%\usepackage{graphics}
\usepackage{longtable}
\usepackage{dcolumn}% Align table columns on decimal point
\usepackage{color}
%\usepackage{underscore}
%\usepackage{html}

\renewcommand{\theequation}{\thesection.\arabic{equation}}
 %  MACROS:

\def\mco{\multicolumn}
\def\epp{\epsilon^{\prime}}
\def\vep{\varepsilon}
\def\ra{\rightarrow}
\def\ppg{\pi^+\pi^-\gamma}
\def\vp{{\bf p}}

\def\be{\begin{equation}}
\def\ee{\end{equation}}
\def\bi{\begin{enumerate}}
\def\ei{\end{enumerate}}
\def\bes{\[}
\def\ees{\]}
\def\bea{\begin{eqnarray}}
\def\eea{\end{eqnarray}}
\def\bar{\begin{eqnarray*}}
\def\ear{\end{eqnarray*}}
\def\CPbar{\hbox{{\rm CP}\hskip-1.80em{/}}}%temp replacement due to no font

\newcommand{\pepc}{{\sc pepc}}
\newcommand{\meno}{{\medskip\noindent}}
\newcommand{\bm}[1]{\mbox{\boldmath$#1$}}
\newcommand{\dbyd}[2]{\frac{\partial #1}{\partial #2}}
\newcommand{\dd}[2]{\frac{d #1}{d #2}}
\newcommand{\ddt}[1]{\frac{d #1}{d t}}
\newcommand{\curl}{\nabla\bm{\times}}
\newcommand{\half}{{1/2}}
\newcommand{\third}{{1/3}}
\newcommand{\quarter}{{1/4}}
\newcommand{\fhalf}{\frac{1}{2}}
\newcommand{\fthreehalf}{\frac{3}{2}}
\newcommand{\fthird}{\frac{1}{3}}
\newcommand{\threehalf}{{3/2}}
\newcommand{\fourthirdpi}{\frac{4\pi}{3}}
\newcommand{\rootpi}{\pi^{\half}}
\newcommand{\Dx}{\Delta x}
\newcommand{\Dt}{\Delta t}
\newcommand{\rij}{\bm{r}_{ij}}

\newcommand{\onemd}{(1-\delta^2)^{\frac{1}{2}}}
\newcommand{\omp}{$\omega_{p}$}
\newcommand{\ompom}{$\omega_{p}/\omega_{0}$}
\newcommand{\cop}{$c/\omega_{p}$}
\newcommand{\Lola}{$L/\lambda$}
\newcommand{\lola}{$L/\lambda$}
\newcommand{\Ilam}{$I\lambda^{2}$}
\newcommand{\Wcmu}{~Wcm$^{-2}\mu$m$^{2}$}
\newcommand{\Wcm}{~Wcm$^{-2}$}
\newcommand{\wcm}{~Wcm$^{-2}$}
\newcommand{\Weight}{$10^{18}$~Wcm$^{-2}$}
\newcommand{\Wnine}{$10^{19}$~Wcm$^{-2}$}
\newcommand{\Jcm}{~Jcm$^{-2}$}
\newcommand{\nonc}{$n_{e}/n_{c}$}
\newcommand{\voc}{$v_{os}/c$}
\newcommand{\vteoc}{$v_{te}/c$}
\newcommand{\vte}{$v_{te}$}
\newcommand{\cmcub}{~cm$^{-3}$}
\newcommand{\gcm}{~gcm$^{-3}$}
\newcommand{\mum}{~$\mu$m}
\newcommand{\cross}{\bm{\times}}
\newcommand{\crossb}{\bm{\times}}
\newcommand{\vcrossb}{$\bm{v\times B}$}


\newcommand{\etal}{\textit{et al.}}
\newcommand{\eq}[1]{Eq.~(\ref{#1})}
\newcommand{\vxB}{\bm{v\times B}}
\newcommand{\jxB}{\bm{j\times B}}
%\newcommand{\deg}{$^\circ$}
\newcommand{\kalpha}{K$_\alpha$}
\newcommand{\eminus}{e$^{-}$}

\newcommand{\upbox}[1]{^{\mbox{\scriptsize\rm #1}}}
\newcommand{\downbox}[1]{_{\mbox{\scriptsize\rm #1}}}
\newcommand{\nph}{^{n+\frac{1}{2}}}
\newcommand{\nmh}{^{n-\frac{1}{2}}}
\newcommand{\np}{^{n+1}}
\newcommand{\nm}{^{n-1}}
\newcommand{\mpo}{^{m+1}}
\newcommand{\mmo}{^{m-1}}
\newcommand{\mph}{^{m+\frac{1}{2}}}
\newcommand{\mmh}{^{m-\frac{1}{2}}}
\newcommand{\ip}{_{i+1}}
\newcommand{\im}{_{i-1}}
\newcommand{\jp}{_{j+1}}
\newcommand{\jm}{_{j-1}}
\newcommand{\iph}{_{i+\frac{1}{2}}}
\newcommand{\imh}{_{i-\frac{1}{2}}}
\newcommand{\jph}{_{j+\frac{1}{2}}}
\newcommand{\jmh}{_{j-\frac{1}{2}}}
\newcommand{\hs}[1]{\hspace{#1 cm}}
\newenvironment{cols}[9]{
\begin{tabbing}
\hspace{.5cm} \= \hspace{3.5cm} \= \hspace{3.5cm} \= \\ \> \texttt{#1} \> \texttt{#2} \> \texttt{#3} \\ \> \texttt{#4} \> \texttt{#5} \> \texttt{#6} \\ \> \texttt{#7} \> \texttt{#8} \> \texttt{#9}}
{\end{tabbing}}

\title{\bf BOPS User Guide\\}
\author{ Version: 3.3 \\ \date{21 April, 2011}
\\ \\ \\
Contributors: \\ \\
Paul Gibbon \\
Anupam Karmakar \\
Bin Qiao \\ \\ \\
J\"ulich Supercomputing Centre,\\
Forschungszentrum J\"ulich GmbH \\
D--52425 J\"ulich, Germany
}


\begin{document}

\maketitle
\pagebreak

\begin{tableofcontents}

\end{tableofcontents}

\pagebreak

\section{Introduction}

BOPS is a one-and-three-halves (1 spatial, 3 velocity coordinates: 1D3V) particle-in-cell code originally created by this author together with Tony Bell in the Plasma Physics Group of Imperial College, London. Based on standard algorithm for a 1D electromagnetic PIC code (Birdsall \& Langdon), BOPS employs a Lorentz transformation, or `boost' along the target surface to mimic the standard 2D, periodic-in-y geometry common to much of the early PIC work on resonance absorption in high-power laser-plasma interactions. 

The technique was first presented at the ECLIM conference in 1990, and later applied to absorption of femtosecond laser pulses on solid targets in PRL 68, 1535 (1992). A longer description of the method including the transformation subtleties can be found later in Section \ref{scaling}.

 While restricting the simulations to a special class of problems -- in which the light, its harmonics and any other scattered modes are reflected in the specular direction only -- the reduction from 2D $\rightarrow$ 1D brings huge savings in computational effort and/or increased spatial and temporal resolution. Not surprisingly, this type of code has become a `workhorse' for high-intensity laser-matter interaction studies, giving relatively easy access to some extremely nonlinear, kinetic plasma phenomena, such as hot electron generation, ion acceleration and high-harmonic generation from solid surfaces.


\section{Prerequisites}

BOPS Version 3.X is a sequential code written in Fortran 90 and requires a full-blooded f90 compiler such as the Intel ifort or GNU's gfortran.  The run scripts provided are designed for generic Unix systems and will work on a Linux PC. 

\subsection*{Linux users}
Unpack the tar file with:
\begin{verbatim}
tar xvfz bopsXX.tar.gz
\end{verbatim}
and cd to the installation directory \texttt{bops}.

\subsection*{Windows users}

The code can be run `under' Windows using an appropriate commercial Fortran developer environment (IDE), but these
platforms are not supported.  (Instructions for successful builds most welcome, however).
Here is a tried-and-tested method which may require a bit of disc space.

\begin{enumerate}
\item First install CYGWIN (www.cygwin.com).  This creates a fully-fledged Unix environment emulator under Windows, 
for which you will need some elementary unix know-how.  
In addition to the default tools/packages offered during the installation you will also need:
\begin{enumerate}
	\item Gnu compilers gcc, g77, g95 etc. (Look in the optional {\it devel} package in Cygwin)
	\item make ({\it devel} package)
	\item vi, emacs ({\it editors} package-- optional, but very handy for quick editing!)
	\item X11 libraries if you want to generate graphics directly under cygwin (eg using gnuplot)
\end{enumerate}
If you forget anything first time, just click on the Cygwin installer icon to locate/update extra packages.
\item Download and unpack the bopsXX.tar.gz file with an archiving tool (eg PowerArchiver: www.powerarchiver.com).
You can put this anywhere, but a convenient location is your 'home' directory under CYGWIN, e.g.:

\begin{verbatim}
C:\cygwin\home\gibbon\bops
\end{verbatim}

\item Open a Cygwin terminal/shell and `cd' to the bops directory.
\end{enumerate}

\section{Installation}
The directory structure resulting from unpacking the tar files should look
like this:

\begin{tabular}{lcl}
\texttt{src/} &	... &	containing the fortran90 source code\\
\texttt{doc/} &	... &	some documentation in html and ps \\
\texttt{run\_scripts/} &	... &	sample scripts for running the code\\
\texttt{tools/} & ... &     postprocessing tools \\
\texttt{example\_plots/} & ... &	sample output graphics
\end{tabular}
	
\medskip
%To compile, first try using the configure script to create a Makefile corresponding closest to your machine/OS (not yet implemented -- coming soon!). 

%\begin{verbatim}
%./configure
%make
%\end{verbatim}
	
Go to the source directory \texttt{src}, adjust and tune the flags in the Makefile to match your machine type (FC=ifort or gfortran etc) and do:

\begin{verbatim}
make
\end{verbatim}

On machines other than a Linux-PC, you may get complaints about the timing routine
\texttt{etime} in the file cputime.f90.  If this happens, edit cputime.f90 and either 
replace etime with something which the compiler knows, or comment it out
altogether - this is not essential to run the code, but handy to know how
long it's going to run for.



\section{Running BOPS}

Once compiled, go to the base (or top) directory and edit one of the examples in
\texttt{run\_scripts} (e.g. resabs).  Change the \$BOPS variable to the directory where the bops.tar file was unpacked (e.g. \$HOME/bops) and the \$RUN variable to where you want the data to be 
placed (e.g.: \texttt{resabs1}).  To run from the base directory, just type

\begin{verbatim}
run_scripts/resabs
\end{verbatim}	
	
This will create a new run directory `resabs1' and start executing the code.
All graphical output etc., will be generated as a series of ASCII files
in the run directory.  Actual graphics are NOT supplied at present,
but there is a postprocessor in the \texttt{tools/gle} directory (od2gle). 
Running the script \texttt{odpp} with the run directory as its argument will create GLE-readable output and .eps or .jpg plots in a separate plots/ subdirectory.
This method of producing graphical output is highly recommended because it gives you an automatically generated, microfiche-like overview
 of the simulation results which you can browse with a simple image viewer.  You can get this program -- Graphics Layout Engine -- for Windows, MAC OS/X and various Linux flavours from:

\begin{verbatim}
http://glx.sourceforge.net
\end{verbatim}

It is an absolute doddle to install and these days even comes with a nice GUI called QGLE.  Further graphics is up to the user - gnuplot or xmgrace will usually 
suffice to get started.  
Some sample plots roughly corresponding to the sample input 
files can be found in the examples directory.

To do a series or parameter study, you might prefer to modify the
script to sit inside a `project' directory and create subdirectories for 
each run.

\subsection{Example scripts}\label{scripts}

\begin{tabular}{ll}
resabs	&	Long scale-length, classical resonance absorption demo\\
snells\_law &	Refractive index transition (underdense plasma)\\
gb\_prl92 &	Vacuum heating demo: steep density gradient, fixed ions\\
foil	&	Thin foil\\
foil+ramp &	Foil + exponential leading ramp\\
foil\_fsmu & 	Foil simulation set up with `experimental units' (fs, microns)\\
hhg	&	High-harmonic generation from plasma surface
\end{tabular}	


\section{Input parameters} 

The variable names below correspond to those appearing in the Fortran namelist file bops.indata read by the code at the start of the run. 
 
\subsection{Target setup}

The first task for the code is to set up the initial plasma conditions -- density profile, 
particle velocity distribution, and so on.   This is done by the routine `parload', and controlled through
various options in the parameter file.  The target type is chosen via two parameters: \texttt{target\_config} and \texttt{inprof}.   The first of these picks the general target class (fixed ions, single species, multi-species), whereas the second parameter determines the shape of the plasma (uniform slab, ramp plus slab, etc.).

\begin{tabular}{lrl}
variable name & value & meaning/selection \\ 
\hline \\
\texttt{target\_config} & 0 & fixed ions \\
& 1 & single ion species \\
& $\geq 2$ & additional ion layers (protons) \\
\texttt{nonc} && $n/n_c$   \\ 
\texttt{xlolam} && $L/\lambda$   \\
\texttt{xm1} &&plasma edge, depending on profile choice \texttt{inprof}\\ 
\texttt{xsol} &&start of 'solid', or max plasma density\\ 
\texttt{xsol2} && RH edge of plasma\\ 
\texttt{xl}&& total grid length \\

\texttt{inprof}: & 1 & uniform profile\\
		&    2 & linear ramp from \texttt{xm1} to \texttt{xl} \\ 
& 3 & linear ramp \texttt{(xm1 - xsol)} + flat top \texttt{(xsol - xl)}\\ 
&   & scalelength \texttt{xlolam} \\ 
& 4 & linear + flat top + trailing ramp \texttt{(xsol - xm2)}\\ 
& 5 & exponential ramp \texttt{xm1} to \texttt{xsol}, scalelength \texttt{xlolam}\\
& 6 & tanh ramp \texttt{(xm1 - xsol)}, scalelength \texttt{xlolam}\\ 
& 7 & foil, thickness \texttt{dfoil}, starting at \texttt{xm1} \\ 
& 8 & 2 uniform layers with densities \\
&& \texttt{nlayer (xm1-xm2)} and  \texttt{n0 (xsol-xsol2)}\\ 
& 57 & foil thickness \texttt{dfoil}, with exponential ramp \\
&& starting at \texttt{xm1}, scalelength \texttt{xlolam}
%\hline
\end{tabular} 

%\epsfig{file=profile_exp,height=2in}
\begin{figure}[tb]
\begin{center}
\includegraphics[totalheight=2in]{figures/profile_lin}
\caption{Setup for linear density profile.}
\end{center}
\label{linear_profile}
\end{figure}
\begin{figure}[tb]
\begin{center}
\includegraphics[totalheight=2in]{figures/profile_exp}
\caption{Setup for exponential density profile.}
\end{center}
\end{figure}

\subsubsection{Multi-species target} 
Two additional multi-ion configurations are currently built in.  The relative proton fraction can be
varied from 0 to 100 \% by choosing an appropriate value for \texttt{rho\_layer}.  For example, setting 
\texttt{nonc=10, rho\_layer=5} will set up an additional layer with 50\% heavy ions and 50\% protons.

\begin{tabular}{lrl}
\hline\\
\texttt{target\_config}& 2 & additional proton/mixed layer on rear of slab (\texttt{inprof=7}) \\ 
\texttt{target\_config}& 3 & additional proton/mixed layer on front of slab \\ 
\texttt{rho\_layer} & & density of layer ($n_p/n_c$) \\
\texttt{x\_layer} & & width of layer \\
\texttt{mpome} & & proton/electron mass ratio \\ 

%\hline
\end{tabular}

\subsubsection{Further plasma parameters} 
\begin{tabular}{lrl}
\hline\\
\texttt{ne}& & number of electrons\\ 
\texttt{ni} & $>0$ & number of ions \\
	    & 0 & ions fixed, with $n_i=n_0$ \\ 
\texttt{miome} & & mass ratio \\ 
\texttt{Z} & & ion charge (1) \\ 
\texttt{amass} & & ion atomic weight (multiplies miome) \\ 
\texttt{Te} & & electron temperature (keV)\\ 
\texttt{Ti} & & ion temperature (keV)\\
%\hline
\end{tabular}

\subsection{Laser} 
\begin{tabular}{lrl}
\hline \\
\texttt{a0}& $>0$ &intensity in \Wcm\ \\ 
\texttt{a0}& $<0$ &pump strength \voc \\ 
\texttt{xlambda} & & wavelength in microns \\
&& (used to calculate \voc\ from \Ilam\ ) \\
\texttt{theta0} && angle of incidence $\theta$ relative to target normal\\ 
\texttt{tpulse} & & pulse duration $t_{fwhm}$ \\
\texttt{tdel} & & pulse delay $t_d$ for Gaussian \\
\texttt{trise} & & rise-time (linear) \\
\texttt{tfall} & & fall-time (linear) \\
\texttt{ilas}:& 1 & uniform sinusoid \\ 
 & 2 & gaussian $I(t) = I_0\exp \{-(t-t_d)^2/t_p^2)\}$;  ($t_p=t_{fwhm}/2\sqrt(\log2)$) \\ 
 & 3 & beat-wave (2-frequency pump) \\
 & 4 & triangular (trise, tfall) \\ 
 & 5 & $I(t) = I_0\sin^2 (\pi t/t_p)$; $t_p=2 t_{fwhm}$\\
\texttt{cpolzn}:&'P' & P-polarized light \\ 
	 &'S' & S-polarized \\ 
  	 & 'C'& C-polarized (circular)\\
% \hline
\end{tabular} 




\subsection{Boundary conditions}
\begin{tabular}{lrl} 
\hline \\
\texttt{ipbc}: & 1 & periodic particles \\ 
	       & 2 & reflective particles\\ 
		& 3 & absorb/reemit at both sides\\ 
		& 4 & absorb ions at LHB, electrons only if charged\\ 
		& 5 & absorb electrons, reflect ions at RHB\\
\texttt{ifbc}: & 1 & periodic fields \\ 
		& 2 & bounded fields -- reflective at solid, RHB
\end{tabular}
 

\subsection{Control, diagnostics}
\begin{tabular}{lrl} 
\hline\\
\texttt{trun} & & total run time \\ 
\texttt{nx}& & number of mesh points\\ 
\texttt{igr} & & frequency of graphical snapshots\\ 
\texttt{itc} & & store for history plots\\ 
\texttt{iout} & & printed output\\ 
\texttt{igmovie} & & time-sequence snapshot frequency\\ 
\texttt{ncyc} & & number of cycles for time-average plots \\
\texttt{igxs} & & only plot every \texttt{igxs} point in 1D plots \\
\texttt{ipskip} & & only plot every \texttt{ipskip} particle in phase-space plots \\ 
\texttt{itsk} & & skip factor in k-space plots \\
\texttt{nsp} & & number of special plots (in splot.f) at specified times \\
\texttt{isp[1:nsp]} & & array giving times for special plots \\
\texttt{iunits} & 0 & time, length normalized to $1/\omega_0$, $c/\omega_0$ (default)\\
 & 1 & time, length normalized to $1/\omega_p$, $c/\omega_p$ \\
 & 2 & time in fs, length in microns \\
\texttt{isubi} & & subcycle ion motion  \\
\texttt{ifreeze} & & freeze-time for ions \\
\texttt{ioboost} & 0 & line plots in lab frame  \\
 & 1 & line plots in boost frame  \\
\texttt{umevmax} & & maximum energy for hot electron spectra  \\
\texttt{uimax} & & maximum energy for ion spectra  \\
\texttt{upmax} & & maximum energy for proton spectra  \\
\texttt{nuav} & & no. timesteps for distribution function average\\
\texttt{vxm, vym} & & maxima for distribution function plots  \\
\texttt{nftem} & & frequency of fourier transform plots\\
\texttt{ift} & & skip factor for FT\\
\texttt{omegm} & & max frequency for FT plots\\
\texttt{ifbin} & & binning for FT plots\\
\texttt{rhotrak} & & tracking density (\texttt{nonc}) \\
\texttt{lrstrt} & (\texttt{.false.}) & restart switch \\
\end{tabular}


\subsection{Particle tracking diagnostics}
\begin{tabular}{lrl} 
\texttt{ntrack} & & number of particle to track \\
\texttt{itropt} & & \\
\texttt{uhot} & & threshold tracking energy \\
\texttt{xpint} & & \\
\texttt{xpstart} & & \\
\texttt{itstart} & & \\
\texttt{itend} & & 
\end{tabular}





\subsection{Units} 
 The default unit system (and the working system of the code) is to normalise time and space variables 
to the laser frequency $\omega_0$ and $c/\omega_0$ respectively. This is achieved by setting \texttt{iunits=0}, the default.  In this case, the other input parameters \texttt{trun}, \texttt{tpulse}, \texttt{trise}, \texttt{tfall}, etc. are all assumed to be in terms of $1/\omega_0$. Given the actual runtime in femtoseconds, the conversion factor is: 
 $$\omega_0 T\downbox{run} = \frac{1.88}{\lambda} T\downbox{run} \mbox{(fs)} $$
where $\lambda$ is the laser wavelength in microns. If you specify the wavelength \texttt{xlambda} in the input file, you will see the 
equivalent elapsed run-time in fs in the printed output during the run. Note that the diagnostic switches \texttt{igr}, \texttt{itc}, \texttt{iout}, \texttt{nftem} 
and \texttt{igmovie} are also normalised to $1/\omega_0$ to make the output control easier. 
All grid length parameters: \texttt{xl}, \texttt{xm1}, \texttt{xsol}, \texttt{xsol2}, \texttt{dfoil}, etc. are normalised to $c/\omega_0$. A grid length of \texttt{xl=12.57} ($4\pi$) is therefore equivalent to 2 vacuum laser wavelengths. 

The other unit choices are \texttt{iunits=1}, which expects the above temporal and spatial input parameters
 in  $1/\omega_p$ and $c/\omega_p$ respectively (sometimes useful if the laser is turned off completely or for
 comparison with other models); and the `experimental' mode \texttt{iunits=2}, where time is in femtoseconds,
 length in microns.  With the latter choice, special care needs to be taken to ensure that the simulation is
set up with numerically stable parameters (particularly that $\lambda_D/\Delta x>0.5$).

The laser intensity is specified (for historical reasons) through the parameter a0, which actually represents the 
normalised pump strength. This gets converted in the code via the relation: 

 $$ a_0 = \sqrt{\frac{ I_0\lambda^2}{1.38\times 10^{18}} } $$

Alternatively, you can specify the pump strength (\voc ) directly by using a negative value for \texttt{a0}, eg. \texttt{a0=-0.5}. 

\subsection{Choosing simulation parameters}
Like most numerical models, PIC codes have to be used with appropriate caution, otherwise they will cough up garbage 
at the earliest opportunity. In particular, it is important to observe certain constraints on spatial
resolution and statistics in order to minimise numerical errors. Each of the sample scripts are set up to produce
 reasonably reliable results, even though the statistics (particle numbers) have been deliberately reduced to a
minimum in the interests of efficiency.  Starting from one of these scripts and scaling up the parameters appropriately should
serve as a good first iteration.  Here is a (far from complete) trouble-shooting checklist in case you suspect something has gone wrong: 
\medskip\\
\begin{tabular}{ll}
Symptom & Remedy \\ \hline
Plasma heats up/expands too quickly & Is the Debye length resolved?  \\
& Should have $\Delta x < 2\lambda_D$  \\ \hline
Excessive noise in electron/ion & Increase particle nos. \texttt{ne, ni} \\
densities or ES field & \\ \hline 
Anomalously high electron energies & Check vacuum regions large enough \\ \hline
Strange energy accounting & Has pulse completely reflected?  \\
 (eg: -ve absorption) & Increase run time \\ \hline
\end{tabular}

\bigskip

\section{Scaling of simulation variables}
\label{scaling}
\begin{figure}[ht]
\begin{center}
\includegraphics[totalheight=2.in]{figures/boost_geom}
\caption{Boost geometry used for simulating oblique incidence interactions in BOPS.}
\end{center}
\label{boost}
\end{figure}
BOPS uses a special technique for modelling oblique-incidence interactions -- a case which normally requires two spatial dimensions, ($x,y$) say, for a target with a gradient in one direction ($x$).  The trick is to perform a Lorentz transformation to a frame in which the wave vector is normally incident, so that the propagation is purely along the target normal --   Fig.\ref{boost}.  Denoting the boost ($S$) frame quantities by primes, the inverse Lorentz transformations for the wave 
frequency and k-vector are:

\bea
\omega' & = & \gamma_0(\omega - v_0k_y) \nonumber \\
\vspace{0.5cm}
k_y' & = & \gamma_0(k_y - \frac{v_0}{c^2}\omega) \nonumber \\
k_{x,z}' & = & k_{x,z} \nonumber 
\eea
%
Since $k_y = k\sin\theta = \omega/c\sin\theta$ and $v_o = c\sin\theta$, we
have:

\bea
k_y' & = & 0 \nonumber \\
\omega' & = & \omega/\gamma_0 \label{freq} \\
k' = \mid {\bf k}'\mid & = & k/\gamma_0 \nonumber
\eea
where $\gamma_0 = 1/\cos\theta$.  For the space and time coordinates, we have (from $S\rightarrow L$)

\bea
t & = & \gamma_0(t' + \frac{v_0}{c^2}y') \nonumber \\
x & = & x' \nonumber  \\
y & = & \gamma_0(y' + v_0 t') \nonumber\\
z & = & z' \label{time} \\
\eea
%
Likewise, noting that $B_x'=0$, the electric and magnetic fields transform as:

$$\left. \begin{array}{rcl}
E_x & = & \gamma_0(E_x' - v_0B_z')  \\
E_y & = & E_y'  \\
B_z & = & \gamma_0(B_z' - \frac{v_0}{c^2}E_x')
\end{array} \right\}  \mbox{p-polarized light,} $$ 
$$\left. \begin{array}{rcl}
E_z & = & \gamma_0 E_z'  \\
B_x & = & \gamma_0v_0E_z'\\
B_y & = & B_y'\label{fields} \\
\end{array} \right\} \mbox{s-polarized light}$$
\begin{equation} \end{equation}
and the density and current for each particle species $\alpha$ (inversely) as:

\bea
\rho_\alpha' & = & \gamma_0(\rho_\alpha - \frac{v_0}{c^2}j_{\alpha y}) \nonumber \\
j_{\alpha y}' & = & \gamma_0(j_{\alpha y} - v_0\rho_\alpha) \nonumber\\
j_{\alpha x,z}' & = & j_{\alpha x,z} \nonumber
\eea
Initially, $j_{ey} = j_{iy} = 0$, so in the boost frame, $(\rho_0)_{e,i}' = \gamma_0(\rho_0)_{e,i}$ and $(j_{y0})_{e,i}' = -(\rho_0)_{e,i}'c\sin\theta$.

In an electromagnetic PIC or Vlasov code, it is convenient to use the standard normalisations \cite{birdsall:book}:

\bea
\tilde{t} & = & \omega t \nonumber \\
(\tilde{x},\tilde{y}) & = & (kx, ky) \nonumber \\
\tilde{{\bf E}} & = & \frac{e{\bf E}}{m\omega c} \nonumber \\
\tilde{{\bf B}} & = & \frac{e{\bf B}}{m\omega} \nonumber \\
\tilde{n}_{e,i} & = & n_{e,i}/n_c \nonumber \\
\tilde{v}_{e,i} & = & v_{e,i}/c \nonumber
\eea

Thus, from Eqs.(\ref{time}) and (\ref{freq}), the normalised time interval
$\omega 't' = \omega t$ is an invariant, as is the wave phase 
$\omega t - {\bf k}\cdot {\bf r}$.  On the other hand, the simulation
grid length $k'x'_l = \gamma_0^{-1}kx_l$ {\em shrinks} with increasing $\theta$.
The scaling of $\omega'$ and $k'$ with $\gamma_0$ and the fact that $\tilde{B}_x'=0$ means that the {\em
normalised} field transformations become:

\bea
\tilde{E}_x & = & \tilde{E}_x' - \tilde{v}_0\tilde{B}_z' \nonumber \\
\tilde{E}_y & = & \frac{\tilde{E}_y'}{\gamma_0} \nonumber \\
\tilde{B}_z & = & \tilde{B}_z' - \tilde{v}_0\tilde{E}_x', \nonumber\\
\tilde{E}_z & = & \tilde{E}_z'  \nonumber\\
\tilde{B}_x & = & \tilde{v}_0\tilde{E}_z' \nonumber \\
\tilde{B}_y & = & \frac{\tilde{B}_y'}{\gamma_0} \label{fields2} 
\eea
%
Note that the critical density in the simulation frame transforms as:

\bea 
\frac{n_c'}{n_c} & = & \frac{\omega '^2}{\omega^2} = \frac{1}{\gamma_0^2} \nonumber 
\eea
%
Hence the initial normalised unperturbed electron density is:

\bea
\tilde{n}_{e0}' & \equiv & \frac{n_e(t=0)'}{n_c'} = \gamma_0^3\tilde{n}_e \nonumber
\eea

Thus, to initialise the particles in the simulation frame, we give them a
charge $q_e' = \gamma_0^2 q_e$, and mass $m_e' = \gamma_0^2 m_e$, which
when combined with the grid length contraction $k'x_l'= kx_l/\gamma_0$, satisfies: $N_e q_e' = -n_e'x_l'$ for the upper density shelf, where $N_e$ is the number of simulation particles. (See Ref.\cite{birdsall:book} for
a discussion on particle loading.)  To initialise the particle momenta,
we first specify the thermal distribution in the lab frame, and then 
perform boosts:

\bea
\tilde{p}_{x,z}' & = & \tilde{p}_{x,z} \nonumber \\
\tilde{p}_y' & = & \gamma_0(\tilde{p}_y - \tilde{v}_0\gamma) \nonumber \\
\gamma' & = & \gamma_0(\gamma - \tilde{v}_0\tilde{p}_y) \label{m_boost} 
\eea

Finally, to launch the EM wave, we must specify its amplitude 
$a_0 =v_{osc}/c$ at the left-hand simulation boundary.  
Since $v_{osc}/c = eE_0/m\omega c$, we have for a $p$-polarized wave:

\bea
a_0' & = & \frac{eE_y'}{m\omega' c} = \gamma_0\frac{eE_y}{m\omega c}
= \frac{eE_0}{m\omega c}\gamma_0\cos\theta = a_0 \label{vosc} 
\eea

One can easily verify this invariance by using Eq.(\ref{fields2}) to 
recover the lab-frame vacuum fields.  According to Eq.(\ref{vosc}), we let $\tilde{E}_y'(x'=0) = \tilde{B}_z'(x'=0) = a_0$, so since $E_x'= 0$, we obtain
$\tilde{E}_x = -v_0a_0 = -a_0\sin\theta$, $\tilde{E}_y = a_0/\gamma_0 = a_0\cos\theta$, and $\tilde{B}_z = a_0$, which is just what we 
expect for a plane wave launched at an angle $\theta$ to the density 
gradient according to Fig.\ref{boost}

\section{Diagnostics}

To interpret the simulation results unambiguously, it is simplest to 
transform the field and particle variables back to the lab frame.
We note in passing, however, that some physical insight and analytical
economy can be 
gained by examining boost frame quantities, so long as care is taken 
in the labeling of sources and fields \cite{ruhl:pop96}.
For the fields, Eq.\ref{fields2} is applied before performing cycle-averages
and Fourier transforms.  For the particles, it is convenient to use the 
inverse transformations of (\ref{m_boost}), from which we can recover the
lab frame velocities, kinetic energies and currents.

\subsection{Absorption rate}
The absorption rate is a little more subtle: since it is useful to be able
to evaluate it in either frame, and we need to take some care over the 
definition.  The boost frame is specially chosen so that the EM waves travel
only along the $x'$-axis.  Therefore, the absorbed wave energy is just the 
 Poynting flux at the left-hand boundary, normalized to the energy
density of the incoming wave:
%
\be
\eta_{wave} = \frac{({F'^+})^2 - ({F'^-})^2}{({F'^+})^2}.
\ee
%
where $F'^+=E'_y+B'_z, F'^-=E_y'-B_z'$.  In the boost frame, the Poynting 
flux normal to the target equals the {\em total} EM flux; in the lab frame,
this is reduced by $\gamma_0$, ie:$P_x = P'\cos\theta = E_yB_z$. 
The cycle-averaged incoming energy is thus 
%
\be 
U_{in} = <P_x> = \frac{1}{2}a_0^2\cos\theta.
\ee
%
This factor is used to obtain the fractional absorption components in thermal energy,
field energy, hot electrons and ions etc.  For example,
%
\be
\eta_{hot} = \frac{U_{hot}}{\int U_{in}dt} \nonumber.
\ee
%
As a consistency check at the end of the simulation, we should have
%
\be
\eta_{wave} = \sum\left(\eta_{hot e} + \eta_{ions} + \eta_{therm} + 
\eta_{field} \right) \nonumber
\ee

\subsection{Electric and magnetic field conversion}
As explained in Section \ref{scaling}, the electric and magnetic fields produced in the output files
are, unless specified otherwise, normalised to
$m\omega_0c/e$ and $m\omega_0/e$ respectively.  To convert back to SI units, the following relations
should be used:
\bea
E &=& \left(\frac{m\omega_0c}{e}\right)\tilde{E} \nonumber \\
  &\simeq & 3.22 \times 10^{12}\lambda_\mu^{-1} [\mbox{V m}^{-1}] \\
& &\\
B &=&  \left(\frac{m\omega_0}{e}\right)\tilde{B} \nonumber \\
 &\simeq & 1.1\times 10^4 \lambda_\mu^{-1} [\mbox{T}] \\
\eea
Similarly, a normalised field intensity $a_0^2=1$ corresponds to 
\bea
I &=& 1.37\times 10^{18} \lambda_\mu^{-2} \mbox{\Wcm} 
\eea
 
\subsection{Conversion factor for hot electrons}
Because BOPS is a 1D code, simulation particles actually represent charge sheets, so one cannot directly
translate their numbers into physical equivalents.  Neverthless, a conversion factor can be estimated
by specifying the laser spot size and wavelength.  The argument goes as follows:

Consider a slab of plasma with very short scale-length (as Fig. \ref{linear_profile} with $L/\lambda=0$).  Since the total charge $Q=N_eq_e$ is conserved, each particle carries a fixed line density:

$$
\Gamma \equiv -\tilde{\rho}_0 = \frac{N_eq_e}{\tilde{L}_p}
$$
where $N_e, \tilde{L}_p, \tilde{\rho}_0, q_e$ are the number of simulation electrons, the plasma length (\texttt{xl-xm1}), normalised density ($\tilde{\rho}_0=n_0/n_c$) and (macro-)charge respectively.  

The number of \emph{physical} electrons contained within a cylinder of length $L_p$ and radius $\sigma_L$ (laser spot size) is just:

$$
N_a = n_0L_p\pi\sigma_L^2
$$
Thus using the default conversion factors
$$L_p = c/\omega_0\tilde{L}_p,$$
 or 
$$\frac{L_p}{\mbox{\mum}} = \frac{\lambda_{\mu}}{2\pi}\tilde{L}_p  
$$
and 
$$
n_0 = \left(\frac{n_0}{n_c}\right)10^{21}\lambda_{\mu}^{-2}\mbox{cm}^{-3} = 10^9\tilde{n}_0\lambda_{\mu}^{-2}\mu m^{-3}$$
we get
\bea
C_s = \frac{N_a}{N_e} &=& 5\times 10^8\lambda_{\mu}^{-1}\sigma_{\mu}^{2}\left(\frac{\tilde{n}_0\tilde{L}_p}{N_e} \right)\nonumber \\
&=& 5\times 10^8\lambda_{\mu}^{-1}\sigma_{\mu}^{2}|q_e|
\eea
This is valid for a plasma slab, but can be refined for more complex profiles by computing the total charge contained in the plasma (a procedure which is performed anyway to get $q_e$ in the first place).
To apply the conversion, just multiply by the number of simulation particles, for example: $$N_{hot}(real) \simeq C_s N_{hot}(sim).$$  

\noindent In the code output (bops.out) $C_s$ is displayed as:
\begin{quote}
\texttt{
Charge conversion factor N/Nsi   3.5244E+07
}
\end{quote}

\section{Output files} 

\begin{tabbing}
 bops.header\hspace{1cm}\=Summary of run parameters and some numerical checks \\
 bops.out\>Continuous run protocol (helpful for debugging)\\ 
 bops.oddata\> List of graphical output files produced during run 
\end{tabbing}

\section{Graphical output} 

The graphical output from the code falls broadly into 6 types of plot: 
\begin{enumerate}
\item Spatial profiles (fields, sources; instantaneous and cycle-averaged) 
\item Particle distribution functions (velocity/energy) 
\item Fourier spectra (time- and space-domains) 
\item Phase space (x-px, x-py, px-py etc) 
\item Time histories (energy diagnostics; absorption; position tracking) 
\item 2D surface plots in x-y plane 
\end{enumerate}

The filenames of the xy plots generated during the run are listed below. Unless otherwise stated, the names consist of 4 letters plus a digit from 00-99 indicating the snapshot number, followed by the suffix .xy. A run lasting 500 time units with graphs produced every 100 units will generate 5 snapshots for each quantity. (At present, the maximum number of snapshots is 9 -- this limitation will be removed in future). The time-histories all have a suffix '0' after the name. 

Assuming you have GLE installed, these plots can be postprocessed with the help of a program in the \texttt{tools} subdirectory.

\begin{verbatim}
cd tools/gle
make od2gle
\end{verbatim}

Make sure the binary generated is in your PATH.  Now go back up to the top directory and run the shell script
\begin{verbatim}
./odpp resabs1
\end{verbatim}
\noindent where \texttt{resabs1} is the name of the run directory you wish to postprocess. The program will scan the file \texttt{foil.id} for plots -- the numbers correspond to the IDs in the tables below -- and attempt to produce series of line graphs for these in \texttt{resabs1/plots}.


\subsection{Spatial profiles} 
 
\noindent Instantaneous\\

\begin{tabular}{rrl}
\hline\\
ID & file & quantity \\
\hline\\
2000 & rhot& net charge density (rhoe+rhoi+rhop) \\ 
2100 & ninc&heavy ion density \\
2200 & ninc&proton density \\
2500 & phsi& electrostatic potential \\
3000 & exsi& field Ex \\
4000 & eysi& field Ey \\
3500 & eysi& field Ez \\
4500 & bzsi& field Bx \\
5500 & bzsi& field By \\
5000 & bzsi& field Bz \\
40000 & tefo& forward going EM wave (boost frame P) \\
40500 & teba& backward going EM wave (boost frame P) \\
44000 & tmfo& forward going EM wave (boost frame S) \\
44500 & tmba& backward going EM wave (boost frame S) \\
41000 & jyel& electron current \\
43000 & jyio& ion current \\
41500 & jtot& net current \\
42500 & ayem& EM vector potential (boost frame only) \\
42000 & azem& EM vector potential (boost frame only) \\
\end{tabular}

\bigskip
\noindent Cycle-averaged\\

\begin{tabular}{rrl}
\hline\\
20000 & exrm& RMS electric field in x-direction \\
21500 & eyrm& RMS electric field in y-direction \\
22000 & bzrm& RMS magnetic field in z-direction \\
21000 & ezrm& RMS electric field in z-direction \\
30000 & byrm& RMS magnetic field in y-direction \\
20500 & exdc& DC electric field in x-direction \\
22500 & bzdc& DC magnetic field in z-direction \\
23500 & phdc& DC electrostatic potential \\
29000 & jyrm& RMS net current in y-direction \\
28500 & jydc& DC net current in y-direction \\
29500 & redc& DC charge density \\
26000 & vxbp& cycle-averaged \vcrossb\ force in x-direction (P)\\
27000 & vxbs& cycle-averaged \vcrossb\ force in x-direction (S)\\
25000 & edoj& RMS \bm{E\cdot J} absorption fraction \\
\end{tabular}

\begin{figure}[ht]
\begin{center}
\includegraphics[totalheight=6in]{figures/vacheat-prof}
\caption{Spatial profiles}
\end{center}
\end{figure}

\subsection{Particle distribution functions} 

\begin{tabular}{rrl}
\hline\\
6000 & fvxe& fe(vx) - electron velocity distribution in vx \\
6200 & fvye& fe(vy) \\
6300 & fvze& fe(vz) \\
6500 & fuep& fe(U) - plasma electron energy spectrum \\
6400 & fuip& fi(U) - heavy ion energy spectrum \\
6600 & fupp& fp(U) - proton energy spectrum \\
6700 & fues& fh(U) - energy distribution of escape hot electrons \\ 
16800 & fuin& energy distribution of re-injected electrons \\
6800 & fhot& hot electron distribution with reinjected thermal electrons subtracted\\ 
7100 & ufue& 1st moment of energy distribution = U*fh(U) \\
7200 & qoqt& cummulative integral form of U*fh(U) \\
& e\_mom\_lhb.dat& electron exit momenta (LHB) \\
& e\_mom\_rhb.dat& electron exit momenta (RHB) \\
& ion\_mom\_lhb.dat& ion exit momenta (LHB) \\
& exit\_energies.dat& electron exit times and energies (RHB)
\end{tabular}

\begin{figure}[ht]
\begin{center}
\includegraphics[totalheight=3in]{figures/vacheat-fhot}
\caption{Fast particle spectra}
\end{center}
\end{figure}

\subsection{Fourier spectra} 
\begin{tabular}{rrl}
\hline\\ 
71000 & tebs& spectrum of reflected EM wave (P-polarized light) \\
70000 & tefs& spectrum of transmitted EM wave (P) \\
76000 & tmbs& spectrum of reflected EM wave (S-polarized light)\\ 
74000 & tmfs& spectrum of transmitted EM wave (S) \\
72000 & escr& frequency-spectrum of electrostatic field at critical density \\
73000 & jycr& frequency-spectrum of current at critical density \\
7300 & uesk& k-spectrum of electrostatic energy density in plasma \\
& ey\_back.t& time-signal of reflected EM E-field \\
& ez\_back.t& time-signal of relfected EM B-field
\end{tabular}

\begin{figure}[ht]
\begin{center}
\includegraphics[totalheight=3in]{figures/vacheat-spec}
\caption{Light (Fourier) spectra}
\end{center}
\end{figure}

\subsection{Phase space (scatter plots)} 
\begin{tabular}{rrl}
\hline\\ 
1000 &pxxe& electron x-momenta vs x\\ 
1500 & pyxe& electron y-momenta vs x \\
1700 & pyxe& electron z-momenta vs x \\
1600 & pxpy& electron y-momenta vs x-momenta \\
1200 & pxxi& ion x-momenta vs x
\end{tabular}

\begin{figure}[ht]
\begin{center}
\includegraphics[totalheight=3in]{figures/vacheat-phase}
\caption{Phase space}
\end{center}
\end{figure}

\subsection{Time histories -- mostly cycle-averaged} 
\begin{tabular}{rl}
\hline\\ 
uinc& incoming electromagnetic wave energy (= normalised laser intensity) \\
ubac& outgoing EM wave energy \\
absr& laser absorption calculated 1-Reflectivity\\ 
uesp& spatially integrated electrostatic wave energy\\ 
ueme& integrated EM energy (TE modes) \\
uthp& thermal energy \\
uthe& electron thermal energy \\
uthi& ion thermal energy \\
urhb& cumulative energy of outgoing electrons (solid boundary)\\ 
ulhb& cumulative energy of outgoing electrons (vacuum boundary) \\
usys& total energy in simulation box \\
utot& total energy including particles lost to boundaries \\
abut& rate of change of total energy dUtot/dt \\
abuh& rate of change of energy of electrons leaving box dUhot/dt \\
abui& rate of change of ion energy \\
nihi& max ion density \\
xcni& position of critical surface 
\end{tabular}

\begin{figure}[ht]
\begin{center}
\includegraphics[totalheight=4in]{figures/vacheat-energy}
\caption{Time histories}
\end{center}
\end{figure}

\subsection{2D surface plots} 
\begin{tabular}{rl}
\hline\\ 

 Bz1.2D& Laser magnetic field \\
 edens1.2D& Electron density \\
 idens1.2D& Ion density \\
 Jy1.2D& Electron current \\
 Jyzoom1.2D& Zoom of electron current\\
\end{tabular}
\begin{figure}[ht]
\begin{center}
\includegraphics[totalheight=1.5in]{figures/eden2d.jpg} \hspace{1cm}
\includegraphics[totalheight=1.5in]{figures/emreflect.jpg}
\caption{2D plots: electron density (left); magnetic field (right)}
\end{center}
\end{figure}





\end{document}
